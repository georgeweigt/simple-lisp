\documentclass[12pt]{article}
\usepackage[margin=2cm]{geometry}
\usepackage{amsmath}
\parindent=0pt
\begin{document}

\begin{center}
\begin{tabular}{lll}
& Math & Lisp
\\[1ex]
Addition & $a+b+c$ & \verb$(sum a b c)$
\\[1ex]
Subtraction & $a-b$ & \verb$(sum a (minus b))$
\\[1ex]
Multiplication & $abc$ & \verb$(product a b c)$
\\[1ex]
Division & $a/b$ & \verb$(product a (inv b))$
\\[1ex]
Power & $a^b$ & \verb$(power a b)$
\\[1ex]
Component & $A^1{}_2$ & \verb$(product A12 (tensor 1 2))$
\end{tabular}
\end{center}

\section*{Symbolic expressions}

Products of sums are expanded.
\begin{verbatim}
? (product a (sum b c))
(sum (product a b) (product a c))

? (power (sum a b) 2)
(sum (power a 2) (power b 2) (product 2 a b))
\end{verbatim}

Sums in an exponent are expanded.
\begin{verbatim}
? (power a (sum b c))
(product (power a b) (power a c))
\end{verbatim}

Vectors, matrices, and tensors are written as sums of components.

\bigskip
The following example computes the inner product of two vectors $A$ and $B$.
\begin{equation*}
A=\begin{pmatrix}A_1\\A_2\end{pmatrix},
\quad
B=\begin{pmatrix}B_1\\B_2\end{pmatrix},
\quad
A\cdot B=A_1B_1+A_2B_2
\end{equation*}
\begin{verbatim}
? (setq A (sum (product A1 (tensor 1)) (product A2 (tensor 2))))
? (setq B (sum (product B1 (tensor 1)) (product B2 (tensor 2))))
? (dot A B)
(sum (product A1 B1) (product A2 B2))
\end{verbatim}

Tensor components can use symbolic indices.
The following example is the same as above except $x$ and $y$ are used for the index names.
\begin{verbatim}
? (setq A (sum (product A1 (tensor x)) (product A2 (tensor y))))
? (setq B (sum (product B1 (tensor x)) (product B2 (tensor y))))
? (dot A B)
(sum (product A1 B1) (product A2 B2))
\end{verbatim}

\end{document}
