\documentclass[12pt]{article}
\usepackage[margin=2cm]{geometry}
\usepackage{amsmath}
\parindent=0pt
\begin{document}

\begin{center}
\begin{tabular}{lll}
& Math & Simple Lisp
\\[1ex]
Addition & $a+b+c$ & \verb$(sum a b c)$
\\[1ex]
Subtraction & $a-b$ & \verb$(sum a (product -1 b))$
\\[1ex]
Multiplication & $abc$ & \verb$(product a b c)$
\\[1ex]
Division & $a/b$ & \verb$(product a (power b -1))$
\\[1ex]
Power & $a^b$ & \verb$(power a b)$
\\[1ex]
Derivative & $\partial f/\partial x$ & \verb$(derivative f x)$
\\[1ex]
Component & $A^1{}_2$ & \verb$(product A12 (tensor 1 2))$
\end{tabular}
\end{center}

\section*{Symbolic expressions}

Products of sums are expanded.
\begin{verbatim}
? (product a (sum b c))
(sum (product a b) (product a c))

? (power (sum a b) 2)
(sum (power a 2) (power b 2) (product 2 a b))
\end{verbatim}

Sums in an exponent are expanded.
\begin{verbatim}
? (power a (sum b c))
(product (power a b) (power a c))
\end{verbatim}

Vectors, matrices, and tensors are written as sums of components.

\bigskip
The following example computes the inner product of two vectors $A$ and $B$.
\begin{equation*}
A=\begin{pmatrix}A_1\\A_2\end{pmatrix},
\quad
B=\begin{pmatrix}B_1\\B_2\end{pmatrix},
\quad
A\cdot B=A_1B_1+A_2B_2
\end{equation*}
\begin{verbatim}
? (setq A (sum (product A1 (tensor 1)) (product A2 (tensor 2))))
? (setq B (sum (product B1 (tensor 1)) (product B2 (tensor 2))))
? (dot A B)
(sum (product A1 B1) (product A2 B2))
\end{verbatim}

Tensor components can use symbolic indices.
The following example is the same as above except $x$ and $y$ are used for the index names.
\begin{verbatim}
? (setq A (sum (product A1 (tensor x)) (product A2 (tensor y))))
? (setq B (sum (product B1 (tensor x)) (product B2 (tensor y))))
? (dot A B)
(sum (product A1 B1) (product A2 B2))
\end{verbatim}

\section*{GR example}

Define the metric tensor.
\begin{equation*}
g_{\mu\nu}=\begin{pmatrix}
-\xi(r) & 0 & 0 & 0
\\
0 & 1/\xi(r) & 0 & 0
\\
0 & 0 & r^2 & 0
\\
0 & 0 & 0 & r^2\sin^2\theta
\end{pmatrix}
\end{equation*}
\begin{verbatim}
(setq gtt (product -1 (xi r)))
(setq grr (power (xi r) -1))
(setq gthetatheta (power r 2))
(setq gphiphi (product (power r 2) (power (sin theta) 2)))

(setq gdd (sum
  (product gtt (tensor t t))
  (product grr (tensor r r))
  (product gthetatheta (tensor theta theta))
  (product gphiphi (tensor phi phi))
))
\end{verbatim}

Calculate $g^{\mu\nu}=(g_{\mu\nu})^{-1}$.
\begin{verbatim}
(setq g (determinant gdd t r theta phi))
(setq guu (product (power g -1) (adjunct gdd t r theta phi)))
\end{verbatim}

Calculate connection coefficients.
\begin{equation*}
\Gamma^\alpha{}_{\beta\gamma}=\frac{1}{2}g^{\alpha\mu}
(g_{\mu\beta,\gamma}+g_{\mu\gamma,\beta}-g_{\beta\gamma,\mu})
\end{equation*}
%
\begin{verbatim}
(define gradient (sum
  (product (derivative arg t) (tensor t))
  (product (derivative arg r) (tensor r))
  (product (derivative arg theta) (tensor theta))
  (product (derivative arg phi) (tensor phi))
))

(setq gddd (gradient gdd))

(setq GAMDDD (product 1/2 (sum
  gddd
  (transpose 2 3 grad)
  (product -1 (transpose 1 2 (transpose 2 3 grad))) ; transpose bc,a to (,a)bc
)))

(setq GAMUDD (dot guu GAMDDD)) ; raise first index
\end{verbatim}

Calculate Riemann tensor.
\begin{equation*}
R^\alpha{}_{\beta\gamma\delta}
=\Gamma^\alpha{}_{\beta\delta,\gamma}
-\Gamma^\alpha{}_{\beta\gamma,\delta}
+\Gamma^\alpha{}_{\mu\gamma}\Gamma^\mu{}_{\beta\delta}
-\Gamma^\alpha{}_{\mu\delta}\Gamma^\mu{}_{\beta\gamma}
\end{equation*}
%
\begin{verbatim}
(setq GAMUDDD (gradient GAMUDD))

(setq GAMGAM (contract 2 4 (product GAMUDD GAMUDD)))

(setq RUDDD (sum
  (transpose 3 4 GAMUDDD)
  (product -1 GAMUDDD)
  (transpose 2 3 GAMGAM)
  (product -1 (transpose 3 4 (transpose 2 3 GAMGAM)))
))
\end{verbatim}

Calculate Ricci tensor.
\begin{equation*}
R_{\mu\nu}=R^\alpha{}_{\mu\alpha\nu}
\end{equation*}
%
\begin{verbatim}
(setq RDD (contract 1 3 RUDDD))
\end{verbatim}

Calculate Ricci scalar.
\begin{equation*}
R=R^\mu{}_\mu
\end{equation*}
%
\begin{verbatim}
(setq R (contract 1 2 (dot guu RDD)))
\end{verbatim}

Calculate Einstein tensor.
\begin{equation*}
G_{\mu\nu}=R_{\mu\nu}-\frac{1}{2}g_{\mu\nu}R
\end{equation*}
%
\begin{verbatim}
(setq GDD (sum RDD (product -1/2 gdd R)))
\end{verbatim}

\end{document}
